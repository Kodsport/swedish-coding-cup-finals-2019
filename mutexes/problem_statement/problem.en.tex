\problemname{Mutexes}
Anna loves coding multithreaded backend services.
In such a service, multiple threads may sometimes need to read and write the same data structures in memory.
To ensure all threads have a consistent view of a single datastructure, one can use so-called \emph{mutexes} to protect access to it.

A mutex is an object that threads can \emph{acquire} and \emph{release}.
When a mutex is acquired, the mutex cannot be acquired again until it is released -- even by the same thread!
If a thread attempts to acquire a mutex it has already acquired, the thread will \emph{deadlock}, waiting for itself to release the mutex.

Anna has written a program that consists of a number of functions.
Each function consists of a list of commands that execute sequentially when the function is called.
The commands are each one of:
\begin{itemize}
  \item acquire the mutex named $X$
  \item release the mutex named $X$
  \item access a data structure protected by the mutex $X$
  \item call another function
\end{itemize}

Anna is not sure if she implemented her mutexes correctly, and she wants your help to verify that three properties hold.
Assuming a function \texttt{main} is called at the beginning of the program, you should check that:
\begin{enumerate}
  \item whenever a data structure protected by a mutex $X$ is to be accessed, the mutex $X$ is currently acquired,
  \item whenever a mutex is to be acquired, the program has not already acquired it (in order to avoid a deadlock),
  \item whenever a mutex is to be released, the program is currently holding it.
\end{enumerate}

\section*{Input}
The first line of the input contains an integer $N$, the number of functions.
This is followed by a description of all the $N$ functions.

The description of a function starts with an integer $M$ and a string $X$, meaning that there is a function named $X$ with $M$ commands.
This is followed by $M$ lines, each containing a command.
The commands will be of the form:
\begin{itemize}
  \item \texttt{acquire X} -- the mutex named $X$ is acquired
  \item \texttt{release X} -- the mutex named $X$ is released
  \item \texttt{access X} -- a data structure that must be protected by the mutex $X$ is accessed
  \item \texttt{call F} -- the function called $F$ is called
\end{itemize}

All functions and mutexes have names between $1-10$ characters in length containing only characters \texttt{a-z}.
No two functions will have the same name, and there will always be a function called \texttt{main}.

It is guaranteed that there is no infinite recursion: a function will never call itself, either directly or through a chain of other functions.

\section*{Output}
If the program is free from errors, output \texttt{a-ok}.

Otherwise, output the first error that occurs during execution. Specifically:
\begin{enumerate}
  \item if a data structure is accessed without the correct mutex being acquired, output \texttt{corruption},
  \item if a deadlock occurs, output \texttt{deadlock},
  \item if a mutex is released without being acquired, output \texttt{error}.
\end{enumerate}

\section*{Scoring}
Your solution will be tested on a set of test groups, each worth a number of points.
To get the points for a test group you need to solve all test cases in the test group.
Your final score will be the maximum score of a single submission.

Let $L$ denote the total number of instructions among all functions, and $K$ the number of mutexes.

\noindent
\begin{tabular}{| l | l | l | l |}
\hline
Group & Points & Constraints \\ \hline
% Simulate everything
1     & TBD    & $L \le 10$ \\ \hline
% DP over position and whether the only mutex is held
2     & TBD    & $L \le 1000$, $K = 1$. It doesn't matter which error you output. \\ \hline
% DP per mutex
3     & TBD    & $L \le 1000$. It doesn't matter which error you output. \\ \hline
% + reconstruction
4     & TBD    & $L \le 1000$ \\ \hline
% BFS
5     & TBD    & $L \le 50\,000$. No \texttt{release} or \texttt{access} commands will be performed. \\ \hline
% Bitsets
6     & TBD    & $L \le 50\,000$ \\ \hline
\end{tabular}
