\problemname{Streaming Services}
Last year, you finally went ahead and became a \emph{cord-cutter}, cancelling your TV subscription to avoid all the annoying ads.
This turned out to not be without problems.
When everyone watched shows live on TV, you could just turn on your TV and always have something to watch.
Nowadays, you would just subscribe to a streaming service and watch TV whenever you want instead.
If you watch TV every day, this may become problematic, since you can accidentally run out of new shows to watch!

A streaming service works like this.
At some specific days, a service will release exactly one new episode of a TV show.
You can watch episodes whenever you want, including on the day of release.
Every day, you want to watch one episode you have never seen before.

To avoid the problem of a single streaming service running out of episodes, you plan to use two streaming services instead.
At any given day, you can buy a subscription of a service that you do not currently subscribe to.
A subscription runs for the following $K$ days, and during each of these days, you can watch episodes on this streaming service.
You may purchase a subscription for the \emph{other} service even if you already have a subscription for the first one.

Given the dates when each service release a new episode for the following $N$ days, determine the total number of subscription periods you need to buy to always have a previously unseen episode to watch on each of the next $N$ days.

\section*{Input}
The first line contains $N$, the number of days you have the release schedule for, and $K$, the number of days each subscription runs for ($1 \le K \le N$).
The next line contains a string of $N$ characters, consisting only of characters \texttt{1}, \texttt{2} or \texttt{B}.

The $i$'th character describes whether a new episode is released on the first service (\texttt{1}), second service (\texttt{2}), or both services (\texttt{B}) on day $i$.

\section*{Output}
Output a single integer; the number of streaming service periods you need to purchase to watch a previously unwatched episode each day.

\section*{Scoring}
Your solution will be tested on a set of test groups, each worth a number of points.
To get the points for a test group you need to solve all test cases in the test group.
Your final score will be the maximum score of a single submission.

\noindent
\begin{tabular}{| l | l | l | l |}
\hline
Group & Points & Constraints \\ \hline
$1$    & $TBD$         & $N \le 1\,000$, $K \le 10$ \\ \hline
$2$    & $TBD$         & $N \le 200\,000$, $K \le 10$ \\ \hline
$3$    & $TBD$         & $N \le 200\,000$, $K \le 200$ \\ \hline
$4$    & $TBD$         & $N \le 200\,000$ \\ \hline
\end{tabular}
